%===========================================================
%                              Choix de track
%===========================================================
% Une des trois options 'parallelisme', 'architecture', 'systeme' 
% doit être utilisée avec le style compas2017
\documentclass[parallelisme]{compas2017}
\usepackage{rotating}
\usepackage{comment}
\usepackage{multirow}
\usepackage{graphicx}
\usepackage{listings}


\graphicspath{{gfx/}}

%===========================================================
%                               Title
%===========================================================

\toappear{1} % Conserver cette ligne pour la version finale

\begin{document}

\title{SCHIaaS : Un laboratoire d'études d'algorithmes par la simulation des clouds.}
\shorttitle{Simuler les clouds}

\author{Julien Gossa, Stéphane Genaud, Luke Bertot}% A voir l'ordre. 
% - Stéphane: moi en dernier.

\address{Université De Strasbourg,\\
Laboratoire ICube - Pôle API - 300 Bd Sébastien Brant\\
67400 Illkirch-Graffenstaden - France\\
julien.gossa@unistra.fr genaud@unistra.fr lbertot@unistra.fr}

\date{\today}

\maketitle

%===========================================================         %
%R\'esum\'e
%===========================================================  
\begin{abstract}

Les clouds ont été intensivement étudiés les dernières années, que ce soit du 
point de vue de l'administrateur qui doit par exemple décider de l'emplacement
des machines virtuelles sur l'infrastructure matérielle, ou du client qui doit 
décider du dimentionnement de sa plateforme virtuelle. Pour ce faire, de nombreux
simulateurs ont vu le jour. Malheureusement, ces derniers se limitent souvent à 
mimer les fonctionnalités d'un cloud. Ils ne proposent pas un cadre complet d'étude
des résultats obtenus, pourtant indispensable à la reproductibiltié des expériences
et la comparaison de solutions différentes. De plus, ils manquent d'ancrage dans la 
réalité, en raison d'une part d'une absence de comparaison à des exécutions réelles, 
et d'autre part d'un ensemble de traces réelles injectables dans les simulations.
Le projet SCHIaaS vise à combler ces deux manques. Nous montrons l'architecture de 
ce \textit{framework}, puis les résultats des expériences de validation.

  \MotsCles{un maximum de 5 mots significatifs, en français, doivent être 
    isolés sous forme de mots-clés.}
\end{abstract}


%=========================================================
\section{Introduction}
%=========================================================

Les  clouds  sont  devenus  ces  dernières  années  une  brique  essentielle  de
l'informatique,   omni-présents  dans   les   architectures   de  nos   systèmes
d'information. La compréhension de leurs  comportements est donc un enjeu majeur
qui a incité les chercheurs à proposer des simulateurs de cloud, essentiellement
pour les clouds de type \textit{Infrastructure-as-a-Service} (IaaS). Ces
simulateurs   reposent   sur   une   modélisation  qu'on   peut   qualifier   de
\textit{bottom-up} : les ressources matérielles sont d'abord modélisées, puis le
comportement des  machines virtuelles sur  ces ressources matérielles,  puis les
applications s'exécutant dans cet environnement. On peut donc distinguer dans 
de nombreux outils de simulation trois composants : 
\begin{itemize}
		\item une spécification de platforme
		\item une spécification de l'application.
		\item un modèle de simulation
\end{itemize}

Le modèle de simulation forme le coeur d'un simulateur. Il décrit l'évolution de
l'état de tous  les constituants du système  simulé en fonction du  temps et des
évènements.   Il  repose  sur  des   modèles  individuels  pour  chacun  de  ces
constituants  (réseau,  processeur,  \ldots).   Etant  donné  la  complexité  du
comportement   des   constituants,  les   modèles   actuels   procèdent  à   des
simplifications  générant  des  erreurs,   accentuées  par  la  combinaison  des
modèles. La précision des modèles utilisés est le principal élément discriminant
entre simulateurs, l'utilisateur  n'ayant généralement que peu de  prise sur cet
aspect%
\footnote{SimGrid  permet  néanmoins  de  choisir  différents  modèles  pour  le
  réseau}.

La spécification de plateforme décrit  l'environnement simulé.  Elle est fournie
par l'utilisateur et décrit les caractéristiques techniques de l'infrastructure,
comme la topologie  d'interconnexion, la puissance des  processeurs, la capacité
des liens réseaux,  etc.  Le niveau de détail de  cette description, qui différe
selon les simulateurs, influence la précision de la simulation.

La  spécification   de  l'application,   fournie  par  l'utilisateur,   est  une
description de la  séquence des évènements à simuler.  Les simulateurs existants
priviligient cette  description sous la forme  d'un programme qui, à  l'aide des
primitives  de  la  bibliothéque  de  simulation,  décrit  le  comportement  des
opération  réelles simulées.   Mais d'autres  mode de  description peuvent  être
proposés,   comme   l'injection  d'une   trace   d'exécution   réelle,  ou   une
représentation abstraite de suite d'opérations. La figure~\ref{fig:sim-features}
donne un aperçu des possibilités de différents simulateurs.


\begin{figure}[hbt]
\begin{tabular}{|c||c|c|c||c|c|c|c|}
	\hline
	%% line 1
	& \multicolumn{3}{c|}{Application} &
	\multicolumn{2}{c|}{CPU}&Réseau&Disque\\
	\cline{2-8}
	%%line 2
	Simulator &(a) 
                  &(b)
                  &(c)
                  &(d)
                  &(e)&
	type & precision max\\
	\hline
	%%line 3
	CloudSim\cite{cloudsim} & & & \bf X &&\bf
	X&store\&forward&seek+transfert\\ \hline
	ICanCloud\cite{icancloud} & & & \bf X &&\bf X&packet& bloc \\ \hline
	GroudSim\cite{groudsim} & & & \bf X &&\bf X&flot& n/a\\\hline
	GreenCloud\cite{greencloud} & & \bf X &&&\bf X&packet& n/a\\ \hline
	SimGrid\cite{simgrid}& \bf X & \bf X & \bf X &\bf X&\bf X& flot/packet &
	seek+transfert \\
	\hline
\end{tabular}
\caption{Principaux simulateurs du domaine et les possibilités de décrire
  l'application : (a) trace d'execution, (b) representation abstraite, (c) simulation
  programmée, ou la performance CPU : (d) delai mesuré, (e) délai
  utilisateur. Le réseau peut être simulé analytiquement (flow) ou avec un
  simulateur externe dédié (packet)}
\label{fig:sim-features}
\end{figure}

Cette description générale montre que la simulation d'une application exécutée
sur un cloud repose sur une combinaison de modèles qui par définition génèrent
une part d'erreur. La quantification de cette erreur est une tâche expérimentale
extrêmement fastidieuse, ce qui explique qu'elle est rarement menée. L'objet
de notre travail est ici de présenter un outil, le \emph{lab}, qui facilite
le travail d'analyse des résultats de simulation. Nous montrons comment cet 
outil peut être utilisé dans le cadre de la simulation d'application cloud
avec le \textit{framework} SCHIaaS qui repose sur SimGrid. Nous présentons 
d'abord cette interface dans la section~\ref{sec:schiaas}, puis blah blah.





%=========================================================
\section{Présentation du \textit{framework}}
%=========================================================

\subsection{SCHIaaS et SimGrid}


SimGrid~\cite{simgrid}  est un  simulateur à  évenements discrets  qui permet  à
l'utilisateur d'exprimer la modélisation  d'une application distribuée à travers
un  programme utilisant  les  primitives  SimGrid qui  décrivent  les phases  de
communications et de calcul des différents processus applicatifs. Ces primitives
sont disponibles à travers différentes interfaces, adaptées aux systèmes étudiés
; il  existe par exemple  des interface  pour les applications  pair-à-pair, MPI
(SMPI), les workflows sous forme de DAG (SimDAG), ou les processus communiquants
de type CSP (MSG).

Le  travail présenté  dans  cet article  repose sur  une  nouvelle interface  de
SimGrid,  nommée SCHIaaS,  dédiée  à  la simulation  de  clouds  IaaS.  Nous  ne
présentons ici que quelques éléments saillants de SCHIaaS mais le lecteur pourra
trouver  une  description  détaillée  sur  le  site  dédié~\cite{schiaas-gforge}.
SCHIaaS  est  un  \textit{framework}   permettant  le  développement  rapide  de
simulateurs de cloud. Il peut être utilisé évaluer des algorithmes au niveau du 
gestionnaire de cloud (appelée aussi \textit{cloud-kit},  par exemple
OpenStack), mais aussi au niveau d'applications clientes d'un cloud.

Développé en JAVA, il offre un ensemble de classes prédéfinies représentant tous
les composants d'un cloud kit. L'utilisateur qui souhaite insérer sa description
du  comportement d'un  ou plusieurs  composants, peut  rapidement greffer  cette
descritpion sous  la forme  d'un programme  qui vient  spécialiser l'une  de ces
classes.  
% ----- inutile maintenant -----
% Le tout  est facilement  instrumentable par  un ensemble  de scripts,
%appelé \emph{lab},  qui permet d'automatiser l'exécution  des simulations, leurs
%observations, et l'analyse des résultats.

\begin{figure}[hbt]
\includegraphics[scale=0.8]{gfx/framework.png}
\caption{ ... Couches Pampers ....}
\label{sec:framework}
\end{figure}

% Voir gfx/framework.txt pour le code source
% Généré avec 
% java -jar /opt/local/share/java/ditaa0_9.jar framework.txt 

La simulation du fonctionnement du  niveau \textit{cloud-kit} est assuré par des
moteurs, exposant les fonctionnalités des clouds au niveau administrateur :
\begin{itemize}
 \item Compute : moteur d'instanciation et de gestion des VM;
 \item Storage : moteur de gestion des stockages;
 \item Network : moteur de gestion de la virtualisation du réseau (à l'état purement abstrait pour l'instant).
\end{itemize}

Chacun de  ces moteurs est  une interface abstraite  assurant l'interopérabilité
entre  eux  ainsi  qu'avec  les   autres  niveaux  du  \textit{framework}.   Des
implémentations de ces  moteurs sont fournies, permettant  d'avoir un simulateur
par défaut pleinement fonctionnel.   Ces implémentations imitent le comportement
d'OpenStack. Un développeur  de fonctionnalités administrateur n'a  donc qu'à se
concentrer sur  la partie  qui l'intéresse, par  exemple différents  systèmes de
stockages.

Les ordonnanceurs de machines virtuelles (VM), qui décident sur quelles machines
physiques  (PM)  seront  déployées  les  VM  demandées  par  l'utilisateur,  ont
également  leur interface  abstraite.  Plusieurs ordonnanceurs  par défaut  sont
fournis,  sur l'exemple  d'Openstack  qui  permet au  choix  d'équilibrer ou  de
consolider les charges  des machines physiques au moyen d'un  poids calculé pour
chacune d'elles.

Un  système abstrait  d'injection de  charge  est également  fourni. Ce  dernier
permet de  contrôler un  nombre de  VM déployées et  les charges  de ces  VM, en
particulier  CPU,   afin  de  faciliter  l'évaluation   des  algorithmes.   Deux
injecteurs  par défaut  sont  fournis. Le  premier est  un  simple injecteur  de
charges sinusoidales  entièrement paramétrables.  Le second injecte  les charges
observées  sur  l'infrastructure  de  Google,   et  rendue  disponible  dans  le
google-cluster-traces.

Ces interfaces ne sont  que celles utilisées dans la suite  de l'article à titre
d'illustration, mais tous les composant d'un \textit{cloud-kit} sont présents
dans SCHIaaS.  
% --- repetition -----
%Chaque module peut être remplacé, étendu, ou
%utilisé en l'état,  ce qui permet le prototypage rapide  de simulateurs de cloud
%adaptés à l'étude de l'utilisateur, sans qu'il  ait à développer plus que ce qui
%l'intéresse.

\subsection{Applications}

SCHIaaS ne peut s'exécuter seul. Dans  la philosophie SimGrid, un simulateur est
une  application exploitant  les fonctionnalités  fournies, mais  qui doit  être
développée et  compilée.  Cette philosophie  est conservée avec SCHIaaS,  qui ne
fait qu'étendre les  fonctionnalités de SimGrid avec celles  disponibles dans un
cloud.

Cependant, deux applications sont fournies par défaut: 
\begin{itemize}
\item \emph{LoadInjector}, qui permet d'injecter  une charge utilisateur sans se
  préoccuper des  applications qui s'exécutent  dans les  VM. Il est  utile pour
  mener des études au niveau administrateur.
\item  \emph{SimSchlouder},  qui   est  la  version  simulée   du  système  réel
  Schlouder\cite{Michon2017}.  Il s'agit d'un  courtier de ressources virtuelles
  capable   de  piloter   de   concert   un  ordonnanceur   de   tâches  et   un
  \textit{cloud-kit}  pour  exécuter  des   calculs  scientifiques  sur  clouds.
  Schlouder et SimSchlouder ont exactement les mêmes entrées/sorties, excepté le
  résulat des  calculs effectués, ce  qui permet de  mener des études  au niveau
  utilisateur de  cloud, sans se  préoccuper du niveau administrateur.  De plus,
  ceci permet de confronter les  performances simulées aux performances réelles,
  et donc de valider la précision de la  simulation. Cela a été fait sur la base
  de  273 exécution  réelles, incluant  8  cas applicatifs  représentatifs et  4
  plateformes  de  cloud  représentatives.  Cette validation  est  en  cours  de
  publication.  Les traces  d'exécution réelles  sont inclues  dans SimSchlouder
  afin de pouvoir servir à la  validation des travaux sur le calcul scientifique
  sur cloud.
\end{itemize}

\subsection{Le Lab}

Le lab est un ensemble de scripts permettant d'automatiser l'exécution des simulations, la collecte des
observations, ainsi que leur analyse. Il permet donc d'exécuter de bout-en-bout l'étude par simulation,
de la définition des différentes simulations, à la production des graphiques.

Au delà de l'aspect pratique et du gain de temps de mise en oeuvre de l'étude, le lab a pour vocation
de ``standardiser'' l'étude. Cette standardisation assure sa reproductibilité, ainsi qu'une comparaison
équitable entre solutions à un même problème.

Mais le lab permet également une approche systématique permetant à l'expérimentateur de ne pas rater 
de phénomène. En effet, il est fréquent de faire un grand lot de simulations, plus d'observer plus
finement celles qui présentent des particularités. Ce faisant, l'expérimentateur exclu les autres 
simulation de ces observations plus précises, à moins de les refaire entièrement. En rendant plus 
pratique la définition des observations directement au niveau du \textit{workflow} de simulations, 
le lab assure que tous les cas seront observés de la même manière, ce qui évite de rater un phénomène,
ou de différencier le traitement des simulations au risque d'arriver à des conclusions abusives.


\section{Cas d'usage concret}

Plaçons-nous dans le cas du problème, parfois appelé \textit{VM Packing}, de 
l'ordonnancement des VMs sur les machines physiques. 
Supposons que l'on dispose d'un algorithme visant à reconsolider 
régulièrement les VMs, c'est à dire à les concentrer sur le moins de PMs 
possibles en les migrant, et que l'on souhaite étudier l'impact du \emph{délais}
entre de ces reconsolidations sur le \emph{nombre de PMs utilisées} et sur le 
\emph{nombre de migrations} nécessaires. 

Pour mener cette étude, les différentes étapes sont :
\begin{enumerate}
 \item la conception du simulateur ;
 \item la description du contexte expérimental des simulations ;
 \item la description et l'exécution les simulations à faire ;
 \item l'analyse les résultats.
\end{enumerate}


\subsection{Conception du simulateur}

Notre \textit{framework} permet de partir sur la base d'un simulateur pleinement 
fonctionnel en sélectionnant les modules par défaut pertinents pour l'études. 
Les développements se limitent alors strictement aux parties spécifiques à 
l'étude.

Dans le contexte de cette étude, le développement concernera seulement 
l'interface \texttt{ComputeScheduler}, qu'il faudra implanter avec l'algorithme 
étudié. Des implantations génériques fournies par défaut facilitent ce 
développement. En l'occurence, il suffira de spécialiser 
\texttt{SimpleReconfigurator}, qui reconfigure le placement des VMs avec un 
\emph{délais} configurable.

Tous les autres modules sont ceux par défaut, et ne nécessitent même pas d'être 
maîtrisés.

Le problème étudié est purement administrateur, seule compte la charge en terme 
de nombre de VMs et de CPU de ces VMs, peu importe les applications qui y sont 
exécutées.Nous allons donc utiliser l'application \emph{LoadInjector}, dont la 
fonction principale est \texttt{loadinjector.SimpleInjection} et qui charge 
simplement un injecteur dans la simulation.

\subsection{Description du contexte expérimental des simulations}

La description du contexte expérimental des simulations se fait au travers de 
fichiers xml, respectant le format standard de SimGrid :

\begin{description}
	\item[platform.xml] décrit l'infrastructure matérielle : machines 
physiques et réseau.
	\item[deploy.xml] décrit les processus.
	\item[cloud.xml] décrit la plateforme de cloud : hôtes pour les VMS,
types d'instance, et ordonnanceurs.
	\item[injector.xml] décrit les injecteurs à utiliser.
\end{description}

Dans notre cas, \texttt{deploy.xml} est inutile, mais il faudra produire 
\texttt{platform.xml} et \texttt{cloud.xml}, ou utiliser les exemples fournis.

Nous allons comparer les résultats avec des délais de reconsolidation de $0$, 
$10$ et $100$. Il faudra donc décliner le fichier de configuration du cloud 
pour ces trois cas : \texttt{cloud-reconsolidator0.xml}, 
\texttt{cloud-reconsolidator10.xml} et \texttt{cloud-reconsolidator100.xml}. 

De plus, il faudra configurer l'injecteur, par exemple en utilisant 
\texttt{SinInjector} qui injecte une charge de calcul sinusoïdale sur un nombre 
de VMs suivant également une sinusoïde.


\subsection{Description et l'exécution les simulations à faire}

L'exécution d'une simulation se fait avec la commande suivante : 
\begin{verbatim}
$ java  loadinjector.SimpleInjection platform.xml deploy.xml cloud.xml
injector.xml
\end{verbatim}

Toutes les expérience du \emph{lab} sont guidées par un fichier de 
configuration. Celui ci contient l'ensemble des information nécessaire aux 
simulations, les variables à observer, et si nécessaire les pré- ou post- 
traitements requis. 

\begin{lstlisting}
# setup
SETUP_DIR ./setup/cmp-scheduler
NEEDED_POST template.R
POST_COMMAND_DATA R -f template.R > R.out

# observations
TU_ARG --count-if used_cores ne 0
TU_ARG --count-if vm:.*:state eq migrating

# simulations
SIM_ARG 1 loadinjector.SimpleInjection
SIM_ARG 2 platform.xml 
SIM_ARG 3 deploy.xml
SIM_ARG 4:reconsolidator0 cloud-reconsolidator10.xml
SIM_ARG 4:reconsolidator10 cloud-reconsolidator10.xml 
SIM_ARG 4:reconsolidator100 cloud-reconsolidator100.xml
SIM_ARG 5 injector.xml
\end{lstlisting}

Les lignes \texttt{$TU\_ARG$} permettent de définir les observations que nous 
souhaitons faire, en l'occurence le nombre de machines dont le nombre de coeur 
utilisé n'est pas nul, ainsi que le nombre de VM en état de migration.

Les lignes \texttt{$SIM\_ARG$} donnent les arguments des simulations à faire. 
En l'occurence, le quatrième argument doit prendre successivement les trois 
fichiers de configuration correspondant au trois ordonnanceurs à comparer. 
Le lab exécute toutes les combinaisons d'arguments possible, ainsi il est 
facile de démultiplier les simulations. Par exemple, l'ajout d'une ligne 
\texttt{$SIM\_ARG 5$} permettrait de faire les mêmes simulations, mais avec 
deux injecteurs différents.

La commande \texttt{./lab.py -p 4 cmp-scheduler.cfg} permet d'exécuter 
cette expérience, avec 4 simulations en parallèle.

\subsubsection{Analyse des résultats}

A la fin des exécution le \emph{lab} extrait des traces de simulation les 
observations demandées et les stocke dans des fichiers séparés, puis exécute 
les commandes stipulées par les lignes \texttt{$POST\_COMMAND$}. En 
l'occurence, il s'agit d'un template R, utilisant une librairie inclue dans le 
lab :

\begin{lstlisting}
pdf('data.pdf')
dfs <- tu_read('.', plotting=TRUE, plotting_state=FALSE)
dev.off()
\end{lstlisting}

Ce dernier va automatiquemet produire un graphique pour chaque observation 
demandée :

\begin{figure}[h]
	\label{output}
	\caption{Graphiques produits automatiquement par le lab.}
	\begin{tabular}{ccc}
\includegraphics[scale=0.30]{reconsolidator0_used_cores_ne_0}&
\includegraphics[scale=0.30]{reconsolidator10_used_cores_ne_0}&
\includegraphics[scale=0.30]{reconsolidator100_used_cores_ne_0}\\
\includegraphics[scale=0.30]{xp_reconsolidator0_slow_vm__state_eq_migrating}&
\includegraphics[scale=0.30]{xp_reconsolidator10_slow_vm__state_eq_migrating}&
\includegraphics[scale=0.30]{xp_reconsolidator100_slow_vm__state_eq_migrating}\\

	\end{tabular}
\end{figure}

On peut observer sur ces six graphiques qu'un intervale de reconsolidation nul 
permet d'optimiser le nombre de machines utilisées, mais au prix de nombreuses 
migrations concurrentes, qu'une intervale de 100 secondes ne présente aucun 
intérêt en terme de d'optimisation, mais qu'un intervale de 10 secondes 
optimise raisonablement la plateforme, tout en évitant toute migration 
concurrente.

Il est imporant de remarquer également que ces graphiques ont les travers 
inévitables des données produites automatiquement : titres abscons, labels 
génériques, et échelles par défaut. L'utilisateur souhaitant raffiner ces 
données devra adapter le fichier \texttt{template.R}.

%=========================================================
\section{Conclusion}
%=========================================================

Le \emph{lab} est un outil pour l'automatisation d'exécution d'expériences
\emph{in silico} qui permet chainé facilement de multiple simulation en faisant
varié les entrées. L'automatisation des simulation et des observation permet non
seulement de s'assurer que rien n'échappe a l'utilisateur mais rend aussi
l'expérience trivialement reproductible. Le \emph{lab} est déjà utilisé au seins
de notre équipes pour la validation de SCHIaaS, en simulant automatiquement
l'ensemble d'une archive de 273 exécution réelle, et pour des simulation de
Monte Carlo, en procédant de manière répétée a 500 simulations d'une même
plateforme avec des taches généré aléatoirement. ???


\bibliography{biblio}

\end{document}



\subsection{SimSchlouder}

L'application \emph{SimSchlouder} permet le prototypage rapide de simulateurs pour évaluer les 
algorithmes d'ordonnancement de VM et de tâches au niveau utilisateurs du cloud. 
Dans ces cas là, l'état de l'infrastructure interne du cloud importe peu, seule compte le service 
rendu à l'utilisateur, isolé sur sa propre plateforme virtuelle. 

Schlouder\cite{Michon2017} est un courtier de ressources virtuelles. Il prend en
entrée la description d'une charge de travail, au format slurm citation???
étendu de quelques informations permettant d'adapter le courtage aux besoins
utilisateurs. Ensuite, Schlouder anaylse la charge de travail soumise, et décide
à l'aide d'algorithmes de dimensionnement (\textit{provisioning}) de déployer un
certains nombres de VM, puis ordonnance les tâches sur ces VMs, et étend les VMs
lorsqu'elles sont devenues inutiles.

SimSchlouder est l'alter-égo de Schlouder, dont l'architecture, les fonctionnalités, le comportement
et les entrées/sorties sont entièrement reproduite dans SCHIaaS. La description d'une charge de 
travail peut donc être soumise indifférement à Schlouder pour une exécution réelle, ou à SimSchlouder
pour une évaluation des choix de courtage par la simulation. Une option de Schlouder permet 
d'ailleurs d'appeler directement SimSchlouder, et d'obtenir le temps de completion et le prix 
de l'exécution demandée en fonction de la plateforme de cloud cible et de la stratégie de réservation
de VM choisie.

Les stratégies d'évaluation des réservations de ressources de cloud ont été intensivement étudiées 
ces dernières années. Cependant, elles sont rarement comparées entre elles, et leur évaluation 
est souvent basée sur des simulations ad-hoc ou inadaptées, et des charges de travail purement artificielles.
SimSchlouder permet de rapidement développer de telles stratégies, et de comparer leurs performances 
à celles fournies par défaut, à savoir ASAP (\textit{As Soon As Possible}, qui vise à minimiser le temps de 
complétion) et AFAP (\textit{As Full As Possible}, qui vise à minimiser les réservations de ressources,
et donc le coût). De plus, cette comparaison peut s'appuyer sur les traces récupérées lors de l'utilisation 
de l'application réelle Schlouder. Pour l'instant, nous mettons à disposition 481 traces d'exécutions
sur deux plateformes : notre cloud privé, homogène, stable et non partagé, et BonFire qui est un cloud 
public hétérogène, instable et partagé, dont les différents sites ont des comportements et des matériels 
assez différents pour qu'on les considère ocmme des clouds différents.
Les charges de travail sont issues de deux applications : OMSSA, qui est un sac de tâches assez courtes 
en protéomique, et MONTAGE, qui est un \textit{workflow} en astronomie ayant des tâches plus imposantes. 

Ainsi, l'utilisation de SimSchlouder permet de ne développer que le coeur d'un stratégie de résevation 
de resources, simplement en implémentant une interface abstraite très simple, pour pouvoir comparer cette 
stratégie aux autres dans des contextes applicatifs et matériels nombreux, représentatifs, et réellement 
éprouvés.

Ces comparaisons nécessitent de très nombreuses simulations, c'est pourquoi nous avons conçu \emph{le lab}.




\begin{enumerate}
	\item Mise en place
		\begin{description}
			\item[SETUP\_DIR] dossier où se trouve tout les fichiers
			\item[POST\_COMMAND\_DATA] commande a exécuter sur les
				observations
			\item[NEEDED\_POST] fichiers requit par 
POST\_COMMAND\_DATA 
		\end{description}
		Il est possible d'exécuter des commandes avant simulation
		(PRE\_COMMAND\_SETUP), ou d'injecter d'autre fichier de commande
		(INCLUDE) eventuelement eux même écrit par la précommande.
	\item Simulations
		\begin{description}
			\item[SIM\_ARG $n$:nom] ensemble des arguments à
				passer au simulateur indexé par leur position
				$n$. Le \emph{lab} exécutera une simulation
				pour chaque combinaison d'arguments possible.
				Pour que index ou plusieurs argument sont
				possible un nom et requit pour différencié les
				simulations.
		\end{description}
		Les l'ensemble des fichiers écrit par chaque simulation sont 
placé
		dans un dossier nommé d'après les argument de la simulation.
	\item Observation
		\begin{description}
			\item[TU\_ARG] ensemble de filtre a appliquer au traces
				de simulation.
		\end{description}
		Les fichier obtenu grâce a TU\_ARGS permettent d'observer les
		variables pertinente de la simulation sans avoir a fouiller
		l'ensemble des traces verbeuses de SCHIaaS. Cette étape est
		cruciale pour l'exploitation automatique des résultats
\end{enumerate}