%===========================================================
%                              Choix de track
%===========================================================
% Une des trois options 'parallelisme', 'architecture', 'systeme' 
% doit être utilisée avec le style compas2017
\documentclass[parallelisme]{compas2017}
\usepackage{rotating}
\usepackage{comment}

\usepackage{graphicx}

\newcommand\vrpath{../../lab/setup/simschlouder/validation-results/}
\graphicspath{{\vrpath}}

%===========================================================
%                               Title
%===========================================================

\toappear{1} % Conserver cette ligne pour la version finale

\begin{document}

\title{SCHIaaS : Un laboratoire d'études d'algorithmes par la simulation des clouds.}
\shorttitle{Simuler les clouds}

\author{Julien Gossa, Stéphane Genaud, Luke Bertot}% A voir l'ordre.

\address{Université De Strasbourg,\\
Laboratoire ICube - Pôle API - 300 Bd Sébastien Brant\\
67400 Illkirch-Graffenstaden - France\\
julien.gossa@unistra.fr genaud@unistra.fr lbertot@unistra.fr}

\date{\today}

\maketitle

%===========================================================         %
%R\'esum\'e
%===========================================================  
\begin{abstract}

Les clouds ont été intensivement étudiés les dernières années, que ce soit du 
point de vue de l'administrateur qui doit par exemple décider de l'emplacement
des machines virtuelles sur l'infrastructure matérielle, ou du client qui doit 
décider du dimentionnement de sa plateforme virtuelle. Pour ce faire, de nombreux
simulateurs ont vu le jour. Malheureusement, ces derniers se limitent souvent à 
mimer les fonctionnalités d'un cloud. Ils ne proposent pas un cadre complet d'étude
des résultats obtenus, pourtant indispensable à la reproductibiltié des expériences
et la comparaison de solutions différentes. De plus, ils manquent d'ancrage dans la 
réalité, en raison d'une part d'une absence de comparaison à des exécutions réelles, 
et d'autre part d'un ensemble de traces réelles injectables dans les simulations.
Le projet SCHIaaS vise à combler ces deux manques. Nous montrons l'architecture de 
ce \textit{framework}, puis les résultats des expériences de validation.

  \MotsCles{un maximum de 5 mots significatifs, en français, doivent être 
    isolés sous forme de mots-clés.}
\end{abstract}


%=========================================================
\section{Introduction}
%=========================================================




%=========================================================
\section{Etat de l'Art}
%=========================================================

EMUSIM combines  emulation and  simulation to extract  information automatically
from  the application  behavior (via  emulation)  and uses  this information  to
generate the  corresponding simulation  model. Such a  simulation model  is then
used  to  build  a simulated  scenario  that  is  closer  to the  actual  target
production  environment   in  terms   computing  resources  available   for  the
application and request patterns.
- EMUSIM operates uniquely with information that is available for customers of public IaaS providers
- Automated Emulation Framework (AEF) [2] for emulation and CloudSim [3] for simulation
- emulation : run the actual software on a subset of the hardware (=hardware model)
- application BoT

%=========================================================
\section{Présentation du \textit{framework}}
%=========================================================

SCHIaaS est un \textit{framework} permettant le développement rapide de simulateurs 
de cloud pour l'évaluation d'algorithmes au niveau du \textit{cloud-kit} et au dessus.
Il est basé sur SimGrid et est développé en JAVA, ce qui permet la spécialisation 
rapide d'un ensemble de classes permettant de greffer les travaux de l'utilisateur au 
niveau qui lui convient, le reste étant assuré par défaut.
Le tout est facilement instrumentable par un ensemble de scripts, appelée \emph{lab},
qui permet d'automatiser l'exécution des simulations, leurs observations, et l'analyse
des résultats.

Schema :

lab (lab.py + schiaas-trace-utils.py )

schiaas (engines + apps)

simgrid (vm + msg)


\subsection{SchIaaS}

SCHIaaS est la sur-couche de SimGrid, qui ajoute par dessus l'interface MSG toutes les 
fonctionnalités d'un cloud. 
Le niveau \textit{cloud-kit} est assuré par mes moteurs, exposant les fonctionnalités 
abstraites des clouds au niveau administrateur :
\begin{itemize}
 \item Compute : moteur d'instanciation et de gestion des VM;
 \item Storage : moteur de gestion des stockages;
 \item Network : moteur de gestion de la virtualisation du réseau (à l'état purement abstrait pour l'instant).
\end{itemize}


%=========================================================
\section{Exemples}
%=========================================================

\subsection{Comparaison d'algorithmes de placement de VM sur PM}

\subsection{Comparaison d'algorithmes d'ordonnancement de tâches sur VM}

\section{Open-science}

\begin{verbatim}
git clone https://git.unistra.fr/gossa/schlouder-traces.git
git clone https://scm.gforge.inria.fr/anonscm/git/schiaas/schiaas.git 
cd schiaas
cmake .
make
cd lab
./lap.py -p2 setup/simschlouder/validation.cfg
cd setup/simschlouder/validation-results
ls
\end{verbatim}


\bibliography{biblio}

\end{document}



