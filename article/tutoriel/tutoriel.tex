%===========================================================
%                              Choix de track
%===========================================================
% Une des trois options 'parallelisme', 'architecture', 'systeme' 
% doit être utilisée avec le style compas2017
\documentclass[parallelisme]{compas2017}
\usepackage{rotating}
\usepackage{comment}
\usepackage{multirow}
\usepackage{graphicx}

\newcommand\vrpath{../../lab/setup/simschlouder/validation-results/}
\graphicspath{{\vrpath}}

%===========================================================
%                               Title
%===========================================================

\toappear{1} % Conserver cette ligne pour la version finale

\begin{document}

\title{SCHIaaS : Un laboratoire d'études d'algorithmes par la simulation des clouds.}
\shorttitle{Simuler les clouds}

\author{Julien Gossa, Stéphane Genaud, Luke Bertot}% A voir l'ordre. 
% - Stéphane: moi en dernier.

\address{Université De Strasbourg,\\
Laboratoire ICube - Pôle API - 300 Bd Sébastien Brant\\
67400 Illkirch-Graffenstaden - France\\
julien.gossa@unistra.fr genaud@unistra.fr lbertot@unistra.fr}

\date{\today}

\maketitle

%===========================================================         %
%R\'esum\'e
%===========================================================  
\begin{abstract}

Les clouds ont été intensivement étudiés les dernières années, que ce soit du 
point de vue de l'administrateur qui doit par exemple décider de l'emplacement
des machines virtuelles sur l'infrastructure matérielle, ou du client qui doit 
décider du dimentionnement de sa plateforme virtuelle. Pour ce faire, de nombreux
simulateurs ont vu le jour. Malheureusement, ces derniers se limitent souvent à 
mimer les fonctionnalités d'un cloud. Ils ne proposent pas un cadre complet d'étude
des résultats obtenus, pourtant indispensable à la reproductibiltié des expériences
et la comparaison de solutions différentes. De plus, ils manquent d'ancrage dans la 
réalité, en raison d'une part d'une absence de comparaison à des exécutions réelles, 
et d'autre part d'un ensemble de traces réelles injectables dans les simulations.
Le projet SCHIaaS vise à combler ces deux manques. Nous montrons l'architecture de 
ce \textit{framework}, puis les résultats des expériences de validation.

  \MotsCles{un maximum de 5 mots significatifs, en français, doivent être 
    isolés sous forme de mots-clés.}
\end{abstract}


%=========================================================
\section{Introduction}
%=========================================================

Les  clouds  sont  devenus  ces  dernières  années  une  brique  essentielle  de
l'informatique,   omni-présents  dans   les   architectures   de  nos   systèmes
d'information. La compréhension de leurs  comportements est donc un enjeu majeur
qui a incité les chercheurs à proposer des simulateurs de cloud, essentiellement
pour les clouds de type \textit{Infrastructure-as-a-Service} (IaaS) . Ces
simulateurs   reposent   sur   une   modélisation  qu'on   peut   qualifier   de
\textit{bottom-up} : les ressources matérielles sont d'abord modélisées, puis le
comportement des  machines virtuelles sur  ces ressources matérielles,  puis les
applications s'exécutant dans cet environnement. On peut donc distinguer dans 
de nombreux outils de simulation trois composants : 
\begin{itemize}
		\item une spécification de platforme
		\item une spécification de l'application.
		\item un modèle de simulation
\end{itemize}

Le modèle de simulation forme le coeur d'un simulateur. Il décrit l'évolution de
l'état de tous  les constituants du système  simulé en fonction du  temps et des
évènements.   Il  repose  sur  des   modèles  individuels  pour  chacun  de  ces
constituants  (réseau,   processeur,  ...).    Etant  donné  la   complexité  du
comportement   des   constituants,  les   modèles   actuels   procèdent  à   des
simplifications  générant  des  erreurs,   accentuées  par  la  combinaison  des
modèles. La précision des modèles utilisés est le principal élément discriminant
entre simulateurs, l'utilisateur  n'ayant généralement que peu de  prise sur cet
aspect%
\footnote{SimGrid  permet  néanmoins de  choisir  différents  modèles pour le réseau}.

La spécification de plateforme décrit l'environement simulé. Elle est fournie
par l'utilisateur et contient un descriptif de la platform sous forme d'objet de
modèle de simulation. Elle est instanciée au début de la simulation et évolue en
fonction de son contenu. Suivant le simulateur la plateform sera plus ou moins
abstraite, certains auron un objet "\emph{Datacenter}" ou d'autre necessiterons
d'instancier séparement chaque machine et câbles ethernet. La granularité de la
spécification de plateforme va généralement de paire avec la précision du
simulateur.

La spécification de l'application, fournie par l'utilisateur est une description
de la séquence des evenements à simuler. Suivant les capacité du simulateur elle
peux être une trace d'execution, une représentation abstraite de suite
d'opérations, ou un programme appelant une bibliothéque de simulation au lieu
d'executé des opération réeles.

\begin{tabular}{|c||c|c|c||c|c|c|c||c|}
	\hline
	%% line 1
	& \multicolumn{3}{c|}{Application} &
	\multicolumn{2}{c|}{CPU}&Réseau&Disque&Platforme\\
	\cline{2-9}
	%%line 2
	Simulator &\rotatebox{90}{trace d'execution\,}&\rotatebox{90}{representation
	abstraite\,}&\rotatebox{90}{simulation programmée\,}&
	\rotatebox{90}{delai mesuré\,}& \rotatebox{90}{délai utilisateur\,}&
	type & precision max&granularité\\
	\hline
	%%line 3
	CloudSim & & & \bf X &&\bf
	X&store/forward&seek+transfert&structure<->equipment\\ \hline
	ICanCloud & & & \bf X &&\bf X&packet& block& equipment \\ \hline
	GroudSim & & & \bf X &&\bf X&flot& n/a & equipment \\\hline
	GreenCloud & & \bf X &&&\bf X&packet& n/a & equipement\\ \hline
	SimGrid & \bf X & \bf X & \bf X &\bf X&\bf X& flot/packet &
	seek+transfert & equipment\\
	\hline
\end{tabular}

%=========================================================
\section{Présentation du \textit{framework}}
%=========================================================

SCHIaaS est un \textit{framework} permettant le développement rapide de simulateurs 
de cloud pour l'évaluation d'algorithmes au niveau du \textit{cloud-kit} et au dessus.
Il est basé sur SimGrid et est développé en JAVA, ce qui permet la spécialisation 
rapide d'un ensemble de classes permettant de greffer les travaux de l'utilisateur au 
niveau qui lui convient, le reste étant assuré par défaut.
Le tout est facilement instrumentable par un ensemble de scripts, appelée \emph{lab},
qui permet d'automatiser l'exécution des simulations, leurs observations, et l'analyse
des résultats.

Schema :

lab (lab.py + schiaas-trace-utils.py )

applications (simschlouder, loadinjector, ...)

schiaas (engines + apps)

simgrid (vm + msg)


\subsection{SCHIaaS}

SCHIaaS est la sur-couche de SimGrid, qui ajoute par dessus l'interface MSG toutes les 
fonctionnalités d'un cloud. 
Le niveau \textit{cloud-kit} est assuré par mes moteurs, exposant les fonctionnalités 
des clouds au niveau administrateur :
\begin{itemize}
 \item Compute : moteur d'instanciation et de gestion des VM;
 \item Storage : moteur de gestion des stockages;
 \item Network : moteur de gestion de la virtualisation du réseau (à l'état purement abstrait pour l'instant).
\end{itemize}

Chacun de ces moteurs est une interface abstraite, qui assure l'interopérabilité entre eux, 
ainsi qu'avec les autres niveaux du \textit{framework}. Des implémentations sont fournies, qui 
permettent d'obtenir un simulateur par défaut pleinement fonctionnel. Ces implémentations 
imitent le comportement d'openstack. Un développeur de fonctionnalités administrateur n'a donc 
qu'à se concentrer sur la partie qui l'intéresse, par exemple différent systèmes de stockages.

De plus, les ordonanceurs de VM, qui décident sur quelle PM seront déployées les VM demandées
par l'utilisateur, ont également leur interface abstraite. Plusieurs ordonnanceurs par défaut
sont fournis, sur l'exemple d'Openstack qui permet au choix d'équilibrer ou de consolider les 
charges des machines physiques au moyen d'un poids calculé pour chacune d'elles. 
Tout moteur compute peut donc réutiliser tout algorithme d'ordonnancement.

Un système abstrait d'injection de charge est également fourni. Ce dernier permet de contrôler 
un nombre de VM déployées et les charges de ces VM, en particulier CPU, afin de faciliter 
l'évaluation des algorithmes. 
Deux injecteurs par défaut sont fournis. Le premier est un simple injecteur de charges sinusoidales
entièrement paramétrables. Le second injecte les charges observées sur l'infrastructure de Google,
et rendue disponible dans le google-cluster-traces.

Ainsi, SCHIaaS fourni, de façon modulaire, tous mes composant d'un \textit{cloud-kit}. 
Chaque module peut être remplacé, étendu, ou utilisé en l'état, ce qui permet le prototypage
rapide de simulateur de cloud adapté à l'étude de l'utilisateur.

\subsection{Applications}

SCHIaaS ne peut s'exécuter seul. Dans la philosophie SimGrid, un simulateur est une application
exploitant les fonctionnalités fournies, mais qui doit être développée et compilée.
Cette philosophie est conservée avec SCHIaaS, qui ne fait qu'étendre les fonctionnalités de 
SimGrid avec celles disponibles dans un cloud.

Cependant, deux applications sont fournies par défaut: \emph{LoadInjector} et \emph{SimSchlouder}.

\subsubsection{LoadInjector}

L'application \emph{LoadInjector} permet le prototypage rapide de simulateurs pour évaluer les 
algorithmes du niveau \textit{cloud-kit}. Dans ces cas là, seule compte la charge en terme de nombre 
de VMs et de CPU de ces VMs, peu importe les applications qui y sont exécutées.

L'exemple typique est celui de l'étude d'algorithme de placement de VM sur PM, il est développé section ???

\subsection{SimSchlouder}

L'application \emph{SimSchlouder} permet le prototypage rapide de simulateurs pour évaluer les 
algorithmes d'ordonnancement de VM et de tâches au niveau utilisateurs du cloud. 
Dans ces cas là, l'état de l'infrastructure interne du cloud importe peu, seule compte le service 
rendu à l'utilisateur, isolé sur sa propre plateforme virtuelle. 

Schlouder citation??? est un courtier de ressources virtuelles. Il prend en entrée la description 
d'une charge de travail, au format slurm citation??? étendu de quelques informations permettant 
d'adapter le courtage aux besoins utilisateurs. Ensuite, Schlouder anaylse la charge de travail 
soumise, et décide à l'aide d'algorithmes de dimensionnement (\textit{provisioning}) de déployer 
un certains nombres de VM, puis ordonnance les tâches sur ces VMs, et étend les VMs lorsqu'elles
sont devenues inutiles.

SimSchlouder est l'alter-égo de Schlouder, dont l'architecture, les fonctionnalités, le comportement
et les entrées/sorties sont entièrement reproduite dans SCHIaaS. La description d'une charge de 
travail peut donc être soumise indifférement à Schlouder pour une exécution réelle, ou à SimSchlouder
pour une évaluation des choix de courtage par la simulation. Une option de Schlouder permet 
d'ailleurs d'appeler directement SimSchlouder, et d'obtenir le temps de completion et le prix 
de l'exécution demandée en fonction de la plateforme de cloud cible et de la stratégie de réservation
de VM choisie.

Les stratégies d'évaluation des réservations de ressources de cloud ont été intensivement étudiées 
ces dernières années. Cependant, elles sont rarement comparées entre elles, et leur évaluation 
est souvent basée sur des simulations ad-hoc ou inadaptées, et des charges de travail purement artificielles.
SimSchlouder permet de rapidement développer de telles stratégies, et de comparer leurs performances 
à celles fournies par défaut, à savoir ASAP (\textit{As Soon As Possible}, qui vise à minimiser le temps de 
complétion) et AFAP (\textit{As Full As Possible}, qui vise à minimiser les réservations de ressources,
et donc le coût). De plus, cette comparaison peut s'appuyer sur les traces récupérées lors de l'utilisation 
de l'application réelle Schlouder. Pour l'instant, nous mettons à disposition 481 traces d'exécutions
sur deux plateformes : notre cloud privé, homogène, stable et non partagé, et BonFire qui est un cloud 
public hétérogène, instable et partagé, dont les différents sites ont des comportements et des matériels 
assez différents pour qu'on les considère ocmme des clouds différents.
Les charges de travail sont issues de deux applications : OMSSA, qui est un sac de tâches assez courtes 
en protéomique, et MONTAGE, qui est un \textit{workflow} en astronomie ayant des tâches plus imposantes. 

Ainsi, l'utilisation de SimSchlouder permet de ne développer que le coeur d'un stratégie de résevation 
de resources, simplement en implémentant une interface abstraite très simple, pour pouvoir comparer cette 
stratégie aux autres dans des contextes applicatifs et matériels nombreux, représentatifs, et réellement 
éprouvés.

Ces comparaisons nécessitent de très nombreuses simulations, c'est pourquoi nous avons conçu \emph{le lab}.

\subsection{Le Lab}

Le lab est un ensemble de scripts permettant d'automatiser l'exécution des simulations, la collecte des
observations, ainsi que leur analyse. Il permet donc d'exécuter de bout-en-bout l'étude par simulation,
de la définition des différentes simulations, à la production des graphiques.

Au delà de l'aspect pratique et du gain de temps de mise en oeuvre de l'étude, le lab a pour vocation
de ``standardiser'' l'étude. Cette standardisation assure sa reproductibilité, ainsi qu'une comparaison
équitable avec d'autres solutions au même problème.

Mais le lab permet également une approche systématique permetant à l'expérimentateur de ne pas rater 
de phénomène. En effet, il est fréquent de faire un grand lot de simulations, plus d'observer plus
finement celles qui présentent des particularités. Ce faisant, l'expérimentateur exclu les autres 
simulation de ces observations plus précises, à moins de les refaire entièrement. En rendant plus 
pratique la définition des observations directement au niveau du \textit{worflow} de simulations, 
le lab assure que tous les cas seront observés de la même manière, ce qui évite de rater un phénomène,
ou de différencier le traitement des simulations au risque d'arriver à des conclusions abusives.


%=========================================================
\section{Exemples}
%=========================================================

\subsection{Comparaison d'algorithmes de placement de VM sur PM}

\subsection{Comparaison d'algorithmes d'ordonnancement de tâches sur VM}

\section{Open-science}

\begin{verbatim}
git clone https://git.unistra.fr/gossa/schlouder-traces.git
git clone https://scm.gforge.inria.fr/anonscm/git/schiaas/schiaas.git 
cd schiaas
cmake .
make
cd lab
./lap.py -p2 setup/simschlouder/validation.cfg
cd setup/simschlouder/validation-results
ls
\end{verbatim}


\bibliography{biblio}

\end{document}



