\documentclass{beamer}
\usepackage[T1]{fontenc}
\usepackage[utf8]{inputenc}
\usepackage{listings}
\usepackage[french]{babel}
\usepackage{graphicx}
\usepackage{tikz}

\graphicspath{{./gfx/}}

% bold font text
\fontseries{bx}\selectfont

%Beamer theme
\usetheme{Madrid}
\setbeamertemplate{navigation symbols}{}%remove navigation symbols
% \AtBeginSection[]{%
% \begin{frame}
% 	\begin{beamercolorbox}[sep=8pt,center,shadow=true,rounded=true]{title}
% 		\usebeamerfont{title}\insertsectionhead\par
% 	\end{beamercolorbox}
% \end{frame}
% }
\newcommand\Wider[2][3em]{%
\makebox[\linewidth][c]{%
  \begin{minipage}{\dimexpr\textwidth+#1\relax}
  \raggedright#2
  \end{minipage}%
  }%
} 



\title[SCHIaaS]{Étude expérimentale par la simulation de clouds avec SCHIaaS}
\author[Luke Bertot]{\underline{Luke Bertot}, Julien Gossa, Stéphane Genaud}
\institute[ICPS]{Équipe ICPS \\
	ICube, Université de Strasbourg---CNRS\\ 
	\{lbertot,gossa,genaud\}@unistra.fr\\
	Pole API, 300Bd Sébastien Brant, CS 10417\\
	F-67412 Illkirch Cedex, France
}
\date{28 Juin 2017}
\titlegraphic{\raisebox{-0.5\height}{\includegraphics[width=1.5cm]{icube-png.png}}\hspace*{1cm}~\raisebox{-0.5\height}{\includegraphics[width=2.5cm]{uds.png}}\hspace*{1cm}~\raisebox{-0.5\height}{\includegraphics[width=1.5cm]{Logo_CNRS.png}}}


\usepackage[backend=bibtex,doi=false,url=false,isbn=false]{biblatex}
\addbibresource{biblio.bib}

\usetikzlibrary{fit,positioning}
\pgfdeclarelayer{background}
\pgfsetlayers{background,main}

\newcommand{\lab}{\texttt{lab}}


\begin{document}
\begin{frame}
	\titlepage{}
\end{frame}

\begin{frame}
	\frametitle{Plan}
	\tableofcontents
\end{frame}

\section{Simulation de cloud}

\begin{frame}
	\frametitle{Simgrid} 
	SimGrid~\footfullcite{simgrid} est un simulateur à évènement discret conçu
	pour l'étude de systèmes distribués développé en C.\bigskip 
	
	SimGrid présente de multiples interfaces pour les applications MPI
	(SMPI), les workflow (SimDAG), ou les processus communicants (MSG).
	\bigskip

	SimGrid est instancié avec une description de la plateforme physique a
	simuler (\texttt{platform.xml}) et lorsque c'est nécessaire une liste de
	tâche a déployer (\texttt{deploy.xml})
\end{frame}

\begin{frame}
	\frametitle{Simulation of Cloud,Hypervisor and IaaS (SCHIaaS)}
	SCHIaaS~\footnotemark est un \emph{framework} permettant
	le développement de simulateur de cloud basée sur SimGrid écrit en JAVA.\bigskip

	SCHIaaS simule l'instanciation et la gestion des VM, la gestion des
	stockage et la visualisation des réseau.\bigskip

	L'architecture modulaire de SCHIaaS permet d'implémenter aisément de
	nouveau comportement au sein de la simulation.\bigskip

	SCHIaaS est initialisé avec la liste des classes a charger et la
	configuration du cloud (\texttt{cloud.xml})

	\footnotetext{\texttt{http://schiaas.gforge.inria.fr}}
\end{frame}

\begin{frame}
	\frametitle{Développer une simulation avec SCHIaaS}
	Pour fonctionner un simulateur SCHIaaS doit :
	\begin{itemize}
		\item Initialiser SimGrid avec la description de la plateforme.
		\item Initialiser SCHIaaS.
		\item Assigner au mon une tache a un ressource de la plateforme.
		\item Lancer l'exécution de la simulation.
	\end{itemize}
	\begin{block}{Tâches SimGrid}
	Une tâche est un code qui déclenche des évènement au sein de la
	simulation. 
	\end{block}
\end{frame}

\begin{frame}[fragile]
	\begin{lstlisting}[language=JAVA]
import org.simgrid.msg
import org.simgird.schiaas
[...]
public static void main(String args[]){
    /*Init Simgrid with platform.xml and deploy.xml*/
        Msg.createEnvironment(args[0]);
        Msg.deployApplication(args[1]);
    /*Init SCHIaaS with cloud.xml*/
        SchIaaS.init(args[2]);
    /* execute the simulation */
        Msg.run();
}
	\end{lstlisting}
\end{frame}

\begin{frame}
	\frametitle{Structure SCHIaaS}
	\Wider[8em]{%
	\begin{tikzpicture}[node distance=5pt,
level/.style={
  font={\sffamily\bfseries\color{black} \fontsize{9pt}{12}\selectfont},
  align=right,
  text width=3cm,
  text height=10pt,
  text depth=4pt},
module1/.style={
  level,
  font={\sffamily\bfseries\color{white} \fontsize{9pt}{12}\selectfont},
  rounded corners,
  text width=9cm+26pt,
  align=center},
module2/.style={
  module1,
  text width=4.5cm+6pt},
module3/.style={
  module1,
  text width=3cm},
moduleset/.style={
  rounded corners,
  fill=lightgray},
]

%\node[level] (Lab) {Lab};
%\node[module2,right=of Lab,fill=blue!90] (labpy) {Pilotage des 
%simulations};
%\node[module2,right=of labpy,fill=blue!90] (TU) {Analyse des 
%observations};
%\begin{pgfonlayer}{background}
%\node[moduleset, fit=(labpy) (TU)] {};
%\end{pgfonlayer}

\node[level] (App) {Applications};
\node[module3,right=of App,fill=blue!85] (LI) {LoadInjector};
\node[module3,right=of LI,fill=blue!85] (SS) {SimSchlouder};
\node[module3,right=of SS,fill=blue!85] () {...};

\node[level,below=of App] (SCHIaaS) {SCHIaaS};
\node[module1,right=of SCHIaaS,fill=blue!75] (API) 
{API de \textit{cloud-kit}};
\node[level,below=of SCHIaaS] (SCHIaaS2) {};
\node[module3,right=of SCHIaaS2,fill=blue!70] (CE) {Compute Engine};
\node[module3,right=of CE,fill=blue!70] (SE) {Storage Engine};
\node[module3,right=of SE,fill=blue!70] (NE) {Network Engine};
\begin{pgfonlayer}{background}
\node[moduleset, fit=(API) (CE) (SE) (NE)] {};
\end{pgfonlayer}

\node[level,below=of App] {SCHIaaS};
\node[module1,right=of SCHIaaS,fill=blue!75]  
{API de \textit{cloud-kit}};
\node[level,below=of SCHIaaS] {};
\node[module3,right=of SCHIaaS2,fill=blue!70] {Compute Engine};
\node[module3,right=of CE,fill=blue!70] {Storage Engine};
\node[module3,right=of SE,fill=blue!70] {Network Engine};

\node[level,below=of SCHIaaS2] (SG) {SimGrid};
\node[module1,right=of SG,fill=blue!50] (SG) {MSG};



\end{tikzpicture}



	}
\end{frame}


\begin{frame}
	\frametitle{Autres Applications}
	\begin{block}{Validation SimSchlouder}
		SimSchlouder est l'implémentation simulée de Schlouder, notre
		système de courtage de cloud. La validation de l'exécution de
		SimSchlouder ce fait en rejouant des trace d'exécution de
		Schlouder dans SimSchlouder. Cette validation présente près de
		1300 simulations automatisées par le \lab.
	\end{block}
	\begin{block}{Simulations de Monte-Carlo}
		La simulation de Monte-Carlo nous permet de prendre en compte la
		variabilité inhérente au cloud en échantillonnant les temps
		d'exécution possible de chacune des tâches. Le \lab{} nous 
		permet d'automatiser l'échantillonnage, les simulation et 
		l'analyse des résultats.
	\end{block}
\end{frame}

\begin{frame}
	\begin{beamercolorbox}[sep=8pt,center,shadow=true,rounded=true]{title}
		\usebeamerfont{title}Merci\par
	\end{beamercolorbox}
\end{frame}

\end{document}

% vim:spell spelllang=fr:
