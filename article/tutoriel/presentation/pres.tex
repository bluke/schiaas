\documentclass{beamer}
\usepackage[T1]{fontenc}
\usepackage[utf8]{inputenc}
\usepackage{listings}
\usepackage[french]{babel}
\usepackage{graphicx}
\usepackage{tikz}
\usepackage{comment}

\graphicspath{{./gfx/}}

% bold font text
\fontseries{bx}\selectfont

%Beamer theme
\usetheme{Madrid}
\setbeamertemplate{navigation symbols}{}%remove navigation symbols
 \AtBeginSection[]{%
 \begin{frame}
 	\begin{beamercolorbox}[sep=8pt,center,shadow=true,rounded=true]{title}
 		\usebeamerfont{title}\insertsectionhead\par
 	\end{beamercolorbox}
 \end{frame}
 }



\title[SCHIaaS]{Étude expérimentale par la simulation de clouds avec SCHIaaS}
\author[Luke Bertot]{\underline{Luke Bertot}, Julien Gossa, Stéphane Genaud}
\institute[ICPS]{Équipe ICPS \\
	ICube, Université de Strasbourg---CNRS\\ 
	\{lbertot,gossa,genaud\}@unistra.fr\\
	Pole API, 300Bd Sébastien Brant, CS 10417\\
	F-67412 Illkirch Cedex, France
}
\date{28 Juin 2017}
\titlegraphic{\raisebox{-0.5\height}{\includegraphics[width=1.5cm]{icube-png.png}}\hspace*{1cm}~\raisebox{-0.5\height}{\includegraphics[width=2.5cm]{uds.png}}\hspace*{1cm}~\raisebox{-0.5\height}{\includegraphics[width=1.5cm]{Logo_CNRS.png}}}


\usepackage[backend=bibtex,doi=false,url=false,isbn=false]{biblatex}
\addbibresource{biblio.bib}

\newcommand{\lab}{\texttt{lab}}


\begin{document}
\begin{frame}
	\titlepage{}
\end{frame}

\begin{frame}
	\frametitle{Plan}
	\tableofcontents
\end{frame}

\section{Simulation de cloud}

\begin{frame}
	\frametitle{Simgrid}
	\resizebox{\textwidth}{!}{\begin{tikzpicture}[x=3cm,y=1cm,
base/.style={
rounded corners,
draw,
outer sep=1mm,
anchor=west,
},
black/.style={
font={\sffamily\bfseries\color{black} \fontsize{9pt}{12}\selectfont},
},
white/.style={
font={\sffamily\bfseries\color{white} \fontsize{9pt}{12}\selectfont},
},
elevel/.style={
align=right,
anchor=east
},
wlevel/.style={
align=left,
anchor=west
},
level/.style={
align=center,
},
w1/.style={
base,
minimum width=2.8cm,
minimum height=0.8cm,
},
w2/.style={
base,
minimum width=5.8cm,
minimum height=0.8cm,
},
w3/.style={
base,
minimum width=8.8cm,
minimum height=0.8cm,
},
h2/.style={
base,
minimum width=2.8cm,
minimum height=1.8cm,
},
bg/.style={
draw,
rounded corners,
fill=gray!15,
anchor=east,
},
]
\node[bg,minimum width=12cm,minimum height=3cm]at(4,1){};
\node[level,font={\sffamily\bfseries\color{black} \fontsize{12pt}{12}\selectfont}]at(2,-1){SimGird};
\node[level,black]at(1,3){Intefaces avec \\code utilisateur};
\node[w1,fill=orange!50,black]at(0,2){MSG};
\node[w1,fill=blue!50,black]at(1,2){SMPI};
\node[elevel,black]at(0,1){Gestion code utilisateur};
\node[w2,fill=green!50,black]at(0,1){SIMIX};
\node[level,black]at(2.5,3){Interface avec\\graphe de t\^ache};
\node[h2,fill=pink!50,black]at(2,1.5){SIMDAG};
\node[elevel,black]at(0,0){Simulation du partage\\des ressources};
\node[w3,black,fill=gray]at(0,0){SURF};
\node[w1,fill=red!50,align=center,font={\sffamily\bfseries\color{black} \fontsize{8pt}{0}\selectfont}]at(3,0){Description\\plateforme};

\end{tikzpicture}}
	SimGrid~\footfullcite{simgrid} est un simulateur à évènement discret 
	conçu pour l'étude de systèmes distribués développé en C.	
\end{frame}

\begin{comment}
	SimGrid est un simulateur à évènement discret conçu
	pour l'étude de systèmes distribués développé en C.\bigskip 
	
	SimGrid présente de multiples interfaces pour les applications MPI
	(SMPI), les workflow (SimDAG), ou les processus communicants (MSG).
	\bigskip

	SimGrid est instancié avec une description de la plateforme physique a
	simuler (\texttt{platform.xml}) et lorsque c'est nécessaire une liste de
	tâche a déployer (\texttt{deploy.xml})
\end{comment}

\begin{frame}
	\frametitle{Simulation of Cloud,Hypervisor and IaaS (SCHIaaS)}
	\resizebox{\textwidth}{!}{\input{gfx/stackbase.tex}}
	SCHIaaS~\footnotemark est un \emph{framework} permettant
	le développement de simulateur de cloud basée sur SimGrid écrit en JAVA.
	\footnotetext{\texttt{http://schiaas.gforge.inria.fr}}
\end{frame}

\begin{frame}
	\frametitle{Simulations : opérateur de cloud}
	\resizebox{\textwidth}{!}{\input{gfx/stackcore.tex}} 
	L'architecture modulaire de SCHIaaS permet d'implémenter aisément de
	nouveau comportement au cloud simulé.
\end{frame}

\begin{frame}
	\frametitle{Simulations : utilisateur de cloud}
	\resizebox{\textwidth}{!}{\begin{tikzpicture}[y=1cm,x=3cm,
base/.style={%
outer sep=1mm,
font={\sffamily\bfseries\color{white} \fontsize{9pt}{12}\selectfont},
minimum width=2.8cm,
minimum height=0.8cm,
rounded corners,
fill=blue,
anchor = west,
draw,
},
level/.style={%
  font={\sffamily\bfseries\color{black} \fontsize{9pt}{12}\selectfont},
  align=right,
  anchor=east},
mod3/.style={%
base,
fill=blue!60,
outer sep=0,
minimum width=9cm,
},
mod2/.style={%
base,
minimum width=5.8cm,
},
top2/.style={%
base,
minimum height=1.8cm,
},
bg/.style={%
draw,
anchor=west,
rounded corners,
fill=gray!50
},
active/.style={
fill=red,
},
passive/.style={
fill=blue!33,
}
]

%SG
\node[level]at(0,0){Simgrid};
\node[mod3]at(0,0){MSG};

%SCHIaaS
\node[level]at(0,1.5){SCHIaaS};
\node[bg,minimum height=2cm,minimum width=9cm]at(0,1.5){};
\node[base]at(0,1){Compute Engine};
\node[base]at(1,1){Storage Engine};
\node[mod2]at(0,2){API de cloud-kit};
\node[top2]at(2,1.5){Tracing};

%Simulator
\node[level]at(0,3){Simulation};
\node[base,active]at(0,3){Load Injector};
\node[base,active]at(1,3){Simschlouder};
\node[base,active]at(2,3){\ldots};
\end{tikzpicture}} 
	Dans la philosophie de SimGrid une simulation est un programme appelant
	les primitives fournie par SimGrid~\ldots{} et SCHIaaS
\end{frame}

\begin{comment}
	SCHIaaS~\footnotemark est un \emph{framework} permettant
	le développement de simulateur de cloud basée sur SimGrid écrit en JAVA.\bigskip

	SCHIaaS simule l'instanciation et la gestion des VM, la gestion des
	stockage et la visualisation des réseau.\bigskip

	L'architecture modulaire de SCHIaaS permet d'implémenter aisément de
	nouveau comportement au sein de la simulation.\bigskip

	SCHIaaS est initialisé avec la liste des classes a charger et la
	configuration du cloud (\texttt{cloud.xml})

	\footnotetext{\texttt{http://schiaas.gforge.inria.fr}}
\end{comment}

\begin{frame}[fragile]
	\frametitle{Developper une simulation avec SCHIaaS}
	Pour fonctionner un simulateur SCHIaaS doit :
	\begin{itemize}
		\item Initialiser SimGrid avec la description de la plateforme.
		\item Initialiser SCHIaaS.
		\item Assigner au moins une tâche a une ressource de la plateforme.
		\item Lancer l'exécution de la simulation.
	\end{itemize}
	\begin{lstlisting}[basicstyle=\footnotesize,language=Java,
	backgroundcolor=\color{gray!10},
	commentstyle=\color{red!90}
	]
import org.simgrid.msg
import org.simgird.schiaas
[...]
public static void main(String args[]){
    /*Init Simgrid with platform.xml and deploy.xml*/
        Msg.createEnvironment(args[0]);
        Msg.deployApplication(args[1]);
    /*Init SCHIaaS with cloud.xml*/
        SchIaaS.init(args[2]);
    /* execute the simulation */
        Msg.run();
}
	\end{lstlisting}
\end{frame}

\section{Automatisation des simulations.}

\begin{frame}
	\frametitle{Enjeux de l'automatisation de simulation.}
\end{frame}

\begin{frame}
	\frametitle{Autres Applications}
	\begin{block}{Validation SimSchlouder}
		SimSchlouder est l'implémentation simulée de Schlouder, notre
		système de courtage de cloud. La validation de l'exécution de
		SimSchlouder se fait en rejouant des traces d'exécution de
		Schlouder dans SimSchlouder. Cette validation présente près de
		1300 simulations automatisées par le \lab.
	\end{block}
	\begin{block}{Simulations de Monte-Carlo}
		La simulation de Monte-Carlo nous permet de prendre en compte la
		variabilité inhérente au cloud en échantillonnant les temps
		d'exécution possibles de chacune des tâches. Le \lab{} nous 
		permet d'automatiser l'échantillonnage, les simulations et 
		l'analyse des résultats.
	\end{block}
\end{frame}

\begin{frame}
	\begin{beamercolorbox}[sep=8pt,center,shadow=true,rounded=true]{title}
		\usebeamerfont{title}Merci\par
	\end{beamercolorbox}
\end{frame}

\end{document}

% vim:spell spelllang=fr:
