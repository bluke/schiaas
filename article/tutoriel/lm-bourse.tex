\documentclass{letter}

\usepackage[T1]{fontenc}
\usepackage[utf8]{inputenc}
\usepackage[francais]{babel}


\address{Luke Bertot \\ Université de Strasbourg \\ \texttt{lbertot@unistra.fr}}
\signature{Luke Bertot}


\begin{document}
\begin{letter}{M. Fabrice Huet \\ Comité d'organisation Compas2017}

\opening{Monsieur,}

A l'occasion de  Compas 2017, je présenterai avec mes  collègues Julien Gossa et
Stéphane Genaud les outils de simulation de clouds au c\oe{}ur de nos travaux, à
travers le tutoriel "simulation de cloud" et de l'article accepté en conférence.
Les outils  que nous avons  développés permettent de structurer  les expériences
impliquant  la  simulation  de   systèmes  informatiques,  en  instrumentant  la
description   d'expériences   ainsi   que   la   collecte   et   l'analyse   des
résultats. Cette approche facilite non seulement la planification et l'exécution
d'expériences, mais aide aussi à la reproductibilité des résultats.

	Compas serait pour moi l'occasion de présenter mes travaux sur les
	simulations cloud par la méthode de Monte Carlo. Celles-ci permettent
	d'étudier le comportement exécution de charge de calcul dans des clouds
	aux capacités exactes inconnues et de faire face à la variabilité
	inhérente aux exécutions sur ce genre de plateforme. Au delà de mes
	travaux actuels Compas me permettra de discuter d'autres problématiques,
	de réfléchir à quel apports mes travaux peuvent avoir dans d'autres
	domaines et surtout comment ces autres problématiques affectent mes
	travaux.




\closing{Dans l'attente de votre réponse, je vous prie d'agréer, Monsieur,
l'expression des mes respectueuses et sincères salutations.}


\end{letter}
\end{document}
