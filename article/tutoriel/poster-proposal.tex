\documentclass[parallelisme]{compas2017}

\title{Proposition de poster pour Compas2017}
\author{Luke Bertot}
\address{Université De Strasbourg,\\
Laboratoire ICube - Pôle API - 300 Bd Sébastien Brant\\
67400 Illkirch-Graffenstaden - France\\
lbertot@unistra.fr }



\begin{document}

\maketitle

%\section*{Information}

%Luke Bertot
%Doctorant à l'Université de Strasbourg
%
%Le poster est basé sur les travaux présententés à Compas2017 lors de la session
%tutoriel \emph{Simulation de Cloud.}\ et par l'article \emph{Méthode pour
%l’étude expérimentale par la simulation de clouds avec SCHIaaS.}\ 

%\section*{Proposition}
\begin{abstract} Les clouds ont été intensivement étudiés les dernières années,
	que ce soit du point de vue de l'administrateur qui doit par exemple
	décider de l'emplacement des machines virtuelles sur l'infrastructure
	matérielle, ou du client qui doit décider du dimensionnement de sa
	plateforme virtuelle. Pour ce faire, de nombreux simulateurs ont vu le
	jour. Malheureusement, ces derniers se limitent souvent à mimer les
	fonctionnalités d'un cloud. Ils manquent d'ancrage dans la réalité, en
	raison d'une part d'une absence de comparaison à des exécutions réelles,
	et d'autre part d'un ensemble de traces réelles injectables dans les
	simulations. De plus, ils ne proposent pas un cadre complet d'étude des
	résultats obtenus pourtant indispensable à la reproductibilité des
	expériences et la comparaison de solutions différentes. Dans
	\emph{Méthode pour l’étude expérimentale par la simulation de clouds
	avec SCHIaaS.}\ présenté à Compas2017 nous montrons avec Julien Gossa et
	Stéphane Genaud comment nos outils forment un framework d'étude par la
	simulation de bout-en-bout, de la conception du simulateur à l'analyse
	des résultats. L'article montre l'utilisation de ces outils pour
	instrumenté une simulation de placement et de consolidation de machine
	virtuelle sur des machine physique. Le poster présentei comment mon
	utilisation de ces outils me permet d'utilisé SCHIaaS, un simulateur a
	evenement discret, comme simulateur stochastique grace à 
	l'implementation de simulations de Monte Carlo.
	
	\MotsCles{simulation; cloud; reproductibilité } \end{abstract}

\end{document}
