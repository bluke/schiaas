



The  problem   of  allocating   cloud  resources   in  performant,   robust  and
energy-efficient ways is  of paramount importance in today's  usage of computing
infrastructures, and a number of research papers have contributed new allocation
techniques to address this issue.  However, the impact of these new ideas on the
design of the  production systems operated by the major  cloud providers remains
questionable.   A pitfall  of research  on IaaS  lies in  the validation  of the
models  and  algorithms  proposed,   which  requires  infrastructures  that  are
difficult  to  set  up  for  individual  researchers.  As  a  consequence,  many
researchers evaluate their work through simulation.

The quality  of the  simulation system  is therefore central  to the  studies of
cloud allocation  systems. We advocate  that a precise assessment  of simulation
should be  carried out against real  execution figures to better  understand the
limits  of simulation  applicability. In  this  paper, we  study the  simulation
accuracy on several use cases ran on  an IaaS clouds. The use cases comprise two
different type  of applications (workflow  and bag-of-tasks), with  several size
instances for  each of them, and  each application is operated  on two different
type of infrastructure (private and public).


We assume an automated process  making the provisioning and scheduling decisions
on  behalf  the user  on  the  real infrastructure.   To  that  purpose, we  use
\emph{Schlouder} \cite{}, a client-side cloud resource broker.  These scheduling
algorithms of Schlouder have been reimplemented in a simulation system, based on
the simulation toolkit SimGrid \cite{simgrid08}.   Our study aims to isolate the
different parameters that  influence the simulation accuracy and  what degree of
divergence between real execution and simulation might be expected in each case.






\begin{comment}
[9] R. N. Calheiros, R. Ranjan, A. Beloglazov, C. A. D. Rose, and R. Buyya, “CloudSim: a toolkit for modeling and simulation of cloud computing environments and evaluation of resource provisioning algo- rithms,” Software: Practice and Experience, vol. 41, no. 1, pp. 23–50,
2011.
[10] D. Kliazovich, P. Bouvry, and S. U. Khan, “GreenCloud: a packet-level simulator of energy-aware cloud computing data centers,” The Journal
of Supercomputing, vol. 62, no. 3, pp. 1263–1283, 2012.
[11] B. Wickremasinghe, R. N. Calheiros, and R. Buyya, “Cloudanalyst: A CloudSim-based visual modeller for analysing cloud computing environments and applications,” in Advanced Information Networking
and Applications (AINA), 2010 24th IEEE International Conference on.
IEEE, 2010, pp. 446–452.
[12] S. K. Garg and R. Buyya, “Networkcloudsim: Modelling parallel applications in cloud simulations,” in Utility and Cloud Computing
(UCC), 2011 Fourth IEEE International Conference on. IEEE, 2011,
pp. 105–113.
[13] M. Tighe, G. Keller, M. Bauer, and H. Lutfiyya, “DCSim: A data centre simulation tool for evaluating dynamic virtualized resource management,” in Network and service management (cnsm), 2012 8th
international conference and 2012 workshop on systems virtualiztion
management (svm), Oct 2012, pp. 385–392.
[14] S. K. S. Gupta, R. Gilbert, A. Banerjee, Z. Abbasi, T. Mukherjee, and G. Varsamopoulos, “GDCSim: A tool for analyzing Green Data Center design and resource management techniques,” in Green Computing
Conference and Workshops (IGCC), 2011 International, July 2011, pp.
1–8.
\end{comment}


 to  study the techniques for  allocating  cloud  resources. in  more  robust,  efficient,  and
ecologically  sustainable  ways. Unfortunately,  the  wide-spread  use of  these
techniques in production systems has, to  date, remained elusive. One reason for
this is that the  state of the art for investigating  these innovations at scale
often  relies solely  on model-driven  simulation.


 Production-grade  cloud software,  however, demands  certainty and  precision for  development and  business
planning  that   only  comes   from  validating  simulation   against  empirical
observation.   In this  work, we  take an  alternative approach  to facilitating
cloud research and engineering in  order to transition innovations to production
deployment faster.  In particular, we present a new methodology that complements
existing  model-driven  simulation   with  platform-specific  and  statistically
trustworthy results.  We simulate systems at  scales and on time frames that are
testable, and  then, based on  the statistical validation of  these simulations,
investigate  scenarios   beyond  those  feasibly  observable   in  practice.  We
demonstrate  the approach  by  developing an  energy-aware  cloud scheduler  and
evaluating it  using production and synthetic  traces in faster than  real time.
Our results show that we can  accurately simulate a production IaaS system, ease
capacity planning, and  expedite the reliable development of  its components and
extensions.
