\subsection{Scheduling}
\textbf{FIXME: chapeau et transition~?}
%----------------------- scheduling approaches ---------------------------------
Compared to earlier distributed computing systems (e.g Grids), the scheduling
problem statement in the cloud must be reformulated because of two fundamental
novelties: firstly the size of the infrastructure can be dynamically scaled, and
secondly as economic cost can be clearly measured, it naturally becomes an
additional objective to the traditional makespan objective, making the problem
multi-objective. An abundance of heuristics have been developed to adapt to the
cloud context, spanning a large variety of assumptions about the setup. The
bi-objective issue is often tackled by combining the cost and makespan metrics
in a single one or fixing one parameter as a constraint (deadline- or budget-
constrained schedules). While it can be argued that several solutions may be
equally interesting trade-offs, it is up to the user to choose. In~\cite{Su13}
is proposed a strategy where VMs belonging to a Pareto front representing the
most cost-efficient resources are used in priority to map jobs. Likewise,
in~\cite{Durillo14} a set of Pareto-optimal solutions are proposed to the user,
and this is also the spirit of our own work. Another discriminating criterion
between approaches is the assumption of an \emph{online} (or
\emph{just-in-time}) or \emph{offline} (or \emph{static}) schedule, on the type
of workload (independent jobs or workflows), the homogeneity of the resources,
or the way they handle the multiple objectives. Offline scheduling represents a
large body of the research often based on the adaptation of popular heuristics
aiming to minimize makespan in Grids. For instance the popular
HEFT~\cite{Zhao2003} heuristic has been revisited in~\cite{LinL11}, which adds
the ability to scale resources depending on whether or not a task can execute by
its estimated finish time, or as extended in~\cite{Li11cost-conscious} to select
slots with a balance between runtime and cost. However, offline scheduling is
often evaluated through simulation while online scheduling is closer to the
practitioners assumptions who evaluate their ideas in actual environments.
Online strategies have been proposed for bags-of-tasks, to address the problem
of appropriately provisioning the platform
(e.g~\cite{MarshallKF10,GenaudG11,DuongLG11,VillegasASI12}). These works share
many commonalities, especially taking into account some important operating
system level details, such as the boot time of a VM or the contention that
occurs if too many VMs are started simultaneously. 